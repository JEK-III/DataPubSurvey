\section*{Figure Legends}
%\begin{figure}[!ht]
%\begin{center}
%%\includegraphics[width=4in]{figure_name.2.png}
%\end{center}
%\caption{
%{\bf Bold the first sentence.}  Rest of figure 2  caption.  Caption 
%should be left justified, as specified by the options to the caption 
%package.
%}
%\label{Figure_label}
%\end{figure}

\begin{figure}[!ht]
\begin{center}
%\includegraphics[width=4in]{figure_name.2.png}
\end{center}
\caption{
{\bf Researchers are largely unfamiliar with US government data sharing and managment policies.}
Respondents based at US institutions self-reported their familiarity with three data-related government funding policies: the Whitehouse OSTP Open Data Initiative ($n=197$), NSF Data Management Plan requirements ($n=197$), and, for biologists only, the NIH data sharing policy ($n=76$).
}
\label{fig:policy_knowledge}
\end{figure}

\begin{figure}[!ht]
\begin{center}
%\includegraphics[width=4in]{figure_name.2.png}
\end{center}
\caption{
{\bf Researchers primarly share data in response to direct contact (e.g. via email).}
Direct contact was the most frequent channel used by respondents to (A. and B.) share their own data and (C.) obtain other's data.
(D.) Respondents documented their shared data through research papers and informal text with roughly equal frequency.
}
\label{fig:sharing}
\end{figure}

\begin{figure}[!ht]
\begin{center}
%\includegraphics[width=4in]{figure_name.2.png}
\end{center}
\caption{
{\bf Formal citation is the preferred method of crediting dataset creators.}
Respondents said that formal citation was (A.) how a dataset creator should be credited and (B.) how they actually credited a dataset creator in the past.
(C.) Most researchers were satisfied with the credit they recieved the last time someone else published using their data.
} 
\label{fig:credit}
\end{figure}



\begin{figure}[!ht]
\begin{center}
%\includegraphics[width=4in]{figure_name.2.png}
\end{center}
\caption{
{\bf Definitions of (A.) data publication and (B.) peer review.}

}
\label{fig:definitions}
\end{figure}

\begin{figure}[!ht]
\begin{center}
%\includegraphics[width=4in]{figure_name.2.png}
\end{center}
\caption{
{\bf Definitions web}

}
\label{fig:definition_web}
\end{figure}

\begin{figure}[!ht]
\begin{center}
%\includegraphics[width=4in]{figure_name.2.png}
\end{center}
\caption{
{\bf Values attached to aspects of peer review and publication.}

}

\label{fig:values}
\end{figure}


\section*{Tables}
\begin{table}[!ht]
\caption{
\bf{Demographics}
\begin{tabular}{|c|c|c|}

 &  & Percent & Count \\
Discipline & Biology & 37 & 91 \\
 & Archaeology & 13 & 31 \\
 & Social science & 13 & 32 \\
 & Environmental science & 11 & 27 \\
 & Physical sciences & 7 & 18 \\
 & Earth science & 5 & 13 \\
 & Computer science & 4 & 11 \\
 & Mathematics & 1 & 3 \\
 & Other & 21 & 21 \\
Role & Principal investigator & 41 & 102 \\
 & Postdoc & 24 & 61 \\
 & Graduate student & 16 & 41 \\
 & Technician & 11 & 28 \\
 & Other & 7 & 17 \\
Highest degree & Doctorate & 76 & 186 \\
 & Masters & 17 & 41 \\
 & Bachelors & 8 & 19 \\
Institution & Academic: research-focused  & 76 & 191 \\
 & Government  & 6 & 14 \\
 & Academic: teaching-focused   & 5 & 13 \\
 & Nonprofit  & 5 & 12 \\
 & Academic: medical school   & 4 & 9 \\
 & Commercial  & 2 & 4 \\
 & Other  & 2 & 6 \\

\begin{flushleft}Demographic breakdown of the 249 researchers whose responses are analyzed here.
\end{flushleft}
\label{tab:demographics}
 \end{table}

\end{document}